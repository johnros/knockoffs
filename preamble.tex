
\usepackage{amsmath}

%% Please use the following statements for
%% managing the text and math fonts for your papers:
\usepackage{times}
%\usepackage[cmbold]{mathtime}
\usepackage{bm}
\usepackage{natbib}
\usepackage{amsfonts}

\usepackage[plain,noend]{algorithm2e}

\makeatletter
\renewcommand{\algocf@captiontext}[2]{#1\algocf@typo. \AlCapFnt{}#2} % text of caption
\renewcommand{\AlTitleFnt}[1]{#1\unskip}% default definition
\def\@algocf@capt@plain{top}
\renewcommand{\algocf@makecaption}[2]{%
  \addtolength{\hsize}{\algomargin}%
  \sbox\@tempboxa{\algocf@captiontext{#1}{#2}}%
  \ifdim\wd\@tempboxa >\hsize%     % if caption is longer than a line
    \hskip .5\algomargin%
    \parbox[t]{\hsize}{\algocf@captiontext{#1}{#2}}% then caption is not centered
  \else%
    \global\@minipagefalse%
    \hbox to\hsize{\box\@tempboxa}% else caption is centered
  \fi%
  \addtolength{\hsize}{-\algomargin}%
}
\makeatother

%%% User-defined macros should be placed here, but keep them to a minimum.
\def\Bka{{\it Biometrika}}
\def\AIC{\textsc{aic}}
\def\T{{ \mathrm{\scriptscriptstyle T} }}
\def\v{{\varepsilon}}
\addtolength\topmargin{35pt}
