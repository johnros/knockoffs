%\documentclass[manuscript]{biometrika}
\documentclass[article,lineno]{biometrika}

\usepackage{amsmath}

%% Please use the following statements for
%% managing the text and math fonts for your papers:
\usepackage{times}
%\usepackage[cmbold]{mathtime}
\usepackage{bm}
\usepackage{natbib}

\usepackage[plain,noend]{algorithm2e}

\makeatletter
\renewcommand{\algocf@captiontext}[2]{#1\algocf@typo. \AlCapFnt{}#2} % text of caption
\renewcommand{\AlTitleFnt}[1]{#1\unskip}% default definition
\def\@algocf@capt@plain{top}
\renewcommand{\algocf@makecaption}[2]{%
  \addtolength{\hsize}{\algomargin}%
  \sbox\@tempboxa{\algocf@captiontext{#1}{#2}}%
  \ifdim\wd\@tempboxa >\hsize%     % if caption is longer than a line
    \hskip .5\algomargin%
    \parbox[t]{\hsize}{\algocf@captiontext{#1}{#2}}% then caption is not centered
  \else%
    \global\@minipagefalse%
    \hbox to\hsize{\box\@tempboxa}% else caption is centered
  \fi%
  \addtolength{\hsize}{-\algomargin}%
}
\makeatother

%%% User-defined macros should be placed here, but keep them to a minimum.
\def\Bka{{\it Biometrika}}
\def\AIC{\textsc{aic}}
\def\T{{ \mathrm{\scriptscriptstyle T} }}
\def\v{{\varepsilon}}
\addtolength\topmargin{35pt}


\begin{document}

\jname{Biometrika}
%% The year, volume, and number are determined on publication
\jyear{2018}
\jvol{yyy}
\jnum{y}
%% The \doi{...} and \accessdate commands are used by the production team
%\doi{10.1093/biomet/asm023}
\accessdate{yyy}

%% These dates are usually set by the production team
\received{{\rm y} yyy {\rm yyyy}}
\revised{{\rm y} yyy {\rm yyyy}}

%\received{January 2017}
%\revised{April 2017}


%% The left and right page headers are defined here:
\markboth{J. D. Rosenblatt \and J.J. Goeman}{Biometrika style}

%% Here are the title, author names and addresses
\title{Discussion of Sesia et al. and the Knockoff Framework}

\author{Jonathan D. Rosenblatt}
\affil{Dept. of Industrial Engineering and Management, \\
	Ben Gurion University of the Negev, Israel.\email{johnros@bgu.ac.il}}

\author{\and Jelle J. Goeman}
\affil{TOOD: Jelle \email{J.J.Goeman@lumc.nl}}

\maketitle

%\begin{abstract}
%TODO
%\end{abstract}


\section{On the problem}
% Variable selection with provable error guarantees. Namely, FDR control.

The authors of \cite{SesiaGenehuntinghidden} set out to design a procedure for variable selection with provable statistical guarantees. 
The \emph{knockoff} algorithm proposed by \cite{SesiaGenehuntinghidden}, provably controls the $FDR$ of conditionally independent variables. 
Denoting with $x$ and $y$ the predictor and outcome variables, respectively.
The \emph{false detection proportion}, a.k.a.\ the \emph{false selection proportion}, is defined as $V/R$ where $R$ is the number of variables selected, and $V$ is the number of such variables where $y|x_{-j}$ is independent of $x_j$. 
The knockoff algorithm of \cite{SesiaGenehuntinghidden} provably control the $FDR:=\mathbb{E}[FDP]$, at some user selected magnitude. 

The fundamental idea of the method is to generate variables that have all the properties of the original $x_j$, only that they are conditionally uncorrelated to $y$. 
These are termed \emph{knockoff} variables. 
The method then proceeds to compute a test statistic that captures the difference in the strength of the dependence of the knockoff, and the original variable. 
This statistic is then compared to it resampling distribution: the distribution over resampled knockoffs. 

Crucially for our comments:
(1) The $FDR$ is an expectation with respect to variability in $x$ and $y$, i.e., a \emph{random design} guarantee. 
(2) The procedure is \emph{model free}, or \emph{non-parametric} in that nothing is assumed on the parametric form of $y|x$. 
(3) The proofs assume full knowledge of $F_x$, i.e., the joint distribution of predictors, marginalized over $y$.
(4) The method aims at good variable selection, not prediction. 

We think of the method in \cite{SesiaGenehuntinghidden} as an adaptation of \cite{CandesPanninggoldmodelX2018} to genome-wide association studies (GWAS).
The differences between the two:
(1) The non-null variables in \cite{CandesPanninggoldmodelX2018} are those that belong to the minimal set that renders all others independent. The non-null variables in \cite{SesiaGenehuntinghidden} are those with non-null partial correlation. 
(2) \cite{CandesPanninggoldmodelX2018} discusses a multivariate Gaussian model, while \cite{SesiaGenehuntinghidden} a hidden Markov model. 
Each paper offers an sampling algorithm for sampling knockoffs from the assumed model. 

The method of \cite{SesiaGenehuntinghidden} is similar in flavor to \cite{BarberControllingfalsediscovery2015}, but \cite{SesiaGenehuntinghidden} is quite more general:
(1) \cite{BarberControllingfalsediscovery2015} assume a linear generative model, so that the null is simply $H_j:\beta_j=0$. 
(2) \cite{BarberControllingfalsediscovery2015} crucially assume $n>p$, and Gaussian distributed errors. 




\section{Other Knockoffs}
The idea of augmenting design matrices with random variables is not new. 
It has been suggested many times, for the purposes of prediction, variable ranking, consistent support recovery, etc. 
Some notable examples include the authors' own \cite{candes2006robust}.
\cite{TusherSignificanceanalysismicroarrays2001} have already proposed the idea of permuting the original variables for FDR control on selected variables.
While intuitive and elegant, their algorithm did not have any provable guarantees. 
Some more algorithms adding ``fake'', ``phony'', ``probes'' or ``pseudo variables'', are reviewed in \cite{GuyonIntroductionVariableFeature2003}.

Perhaps the most similar work is that of \cite{WuControllingVariableSelection2007}, which not only propose adding `pseudo-variables'' for the purpose of estimating the variable selection FDR, but also require two conditions very similar to the knockoff conditions. 
\cite{WuControllingVariableSelection2007} require that:
(A1) ``real unimportant variables and phony unimportant variables have the same probability of being selected on average'', and (A2)``real important variables have the same probability of being selected whether or not phony variables are present''.
These two conditions cannot be satisfied, but they are clearly related to the \emph{pairwise exchangeability} and \emph{nullity condition} in \cite{SesiaGenehuntinghidden} and \cite{CandesPanninggoldmodelX2018}.

The impossibility to satisfy A1 and A2 was already observed by \cite{WuControllingVariableSelection2007}. 
One may thus view the two knockoff conditions as a satisfiable version of A1 and A2.
To the credit of \cite{WuControllingVariableSelection2007} we quote their insights, which already hint at what will be later formalized in the knockoff conditions:
``Permutation produces pseudovariables that when appended to the real data create what
are essentially matched pairs. To each real variable there corresponds a pseudo variable with identical sample moments and also with preservation of correlations''.








\section{Other Variable Selection Methods}
% stability selection, SURE screening, Banjamini-Gavrilov, BOLASO, hard thresholding, adaptive lasso, …




\section{Permutation Testing and Symmetries}



\section{Future Research}
% robustness to $$F_x$$, robustness to true correlations, best statistic for power, algorithms to generate knockoffs, statistics for more than one knockoff,…


\section{On the hypotheses}
% How does knockoff resolve identifiability? Why hypotheses changing between papers? And link to other methods (stability selection, SURE screening, Banjamini-Gavrilov, BOLASO, hard thresholding, adaptive lasso, …).

\section{On the problem setup}
% Random versus fixed design.
% Adequancy for GWAS?



\section*{Acknowledgement}
The authors thank Prof. Yaakov Ritov, Dr. Aldo Solari, Dr. Livio Finos, and ... for fruitful discussions leading to this manuscript. 




\bibliographystyle{biometrika}
\bibliography{paper-ref}



\end{document}
