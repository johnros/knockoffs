%\documentclass[manuscript]{biometrika}
\documentclass[article,lineno]{biometrika}

\usepackage{amsmath}

%% Please use the following statements for
%% managing the text and math fonts for your papers:
\usepackage{times}
%\usepackage[cmbold]{mathtime}
\usepackage{bm}
\usepackage{natbib}

\usepackage[plain,noend]{algorithm2e}

\makeatletter
\renewcommand{\algocf@captiontext}[2]{#1\algocf@typo. \AlCapFnt{}#2} % text of caption
\renewcommand{\AlTitleFnt}[1]{#1\unskip}% default definition
\def\@algocf@capt@plain{top}
\renewcommand{\algocf@makecaption}[2]{%
  \addtolength{\hsize}{\algomargin}%
  \sbox\@tempboxa{\algocf@captiontext{#1}{#2}}%
  \ifdim\wd\@tempboxa >\hsize%     % if caption is longer than a line
    \hskip .5\algomargin%
    \parbox[t]{\hsize}{\algocf@captiontext{#1}{#2}}% then caption is not centered
  \else%
    \global\@minipagefalse%
    \hbox to\hsize{\box\@tempboxa}% else caption is centered
  \fi%
  \addtolength{\hsize}{-\algomargin}%
}
\makeatother

%%% User-defined macros should be placed here, but keep them to a minimum.
\def\Bka{{\it Biometrika}}
\def\AIC{\textsc{aic}}
\def\T{{ \mathrm{\scriptscriptstyle T} }}
\def\v{{\varepsilon}}
\addtolength\topmargin{35pt}


\begin{document}

\jname{Biometrika}
%% The year, volume, and number are determined on publication
\jyear{2018}
\jvol{yyy}
\jnum{y}
%% The \doi{...} and \accessdate commands are used by the production team
%\doi{10.1093/biomet/asm023}
\accessdate{yyy}

%% These dates are usually set by the production team
\received{{\rm y} yyy {\rm yyyy}}
\revised{{\rm y} yyy {\rm yyyy}}

%\received{January 2017}
%\revised{April 2017}


%% The left and right page headers are defined here:
\markboth{J. D. Rosenblatt \and J.J. Goeman}{Biometrika style}

%% Here are the title, author names and addresses
\title{Discussion of Sesia et al. and the Knockoff Framework}

\author{Jonathan D. Rosenblatt}
\affil{Dept. of Industrial Engineering and Management, \\
	Ben Gurion University of the Negev, Israel.\email{johnros@bgu.ac.il}}

\author{\and Jelle J. Goeman}
\affil{Department of Biomedical Data Sciences, 
Leiden University Medical Center, The Netherlands.\email{J.J.Goeman@lumc.nl}}

\maketitle

%\begin{abstract}
%TODO
%\end{abstract}


\section{On the Motivation}
% Variable selection with provable error guarantees. Namely, FDR control.

The authors of \cite{SesiaGenehuntinghidden} set out to design a procedure for variable selection with provable statistical guarantees.
The \emph{knockoff} algorithm proposed by \cite{SesiaGenehuntinghidden}, provably controls the $FDR$ of conditionally independent variables.
Denoting with $x$ and $y$ the predictor and outcome variables, respectively.
The \emph{false discovery proportion}, a.k.a.\ the \emph{false selection proportion}, is defined as $FDP:=V/R$ where $R$ is the number of variables selected, and $V$ is the number of falsely selected.
A false selection defined by \cite{SesiaGenehuntinghidden} to be a selected $x_j$ where $y|x_{-j}$ is independent of $x_j$.
The knockoff algorithm of \cite{SesiaGenehuntinghidden} provably control the $FDR:=\mathbb{E}[FDP]$, at some user selected magnitude.

[TODO: consistent notation $x$ or $X$?]

The fundamental idea of the method is to generate variables that have all the properties of the original $x_j$, only that they are conditionally uncorrelated to $y$.
These are termed \emph{knockoff} variables.
The method then proceeds to compute a test statistic that captures the difference between the dependence of $y$ to $x_j$ and to its knockoff.
[TODO: how is a variable actually selected?]

Crucially for our discussion:
(1) The $FDR$ is an expectation with respect to variability in $x$ and $y$, i.e., a \emph{random design} guarantee.
(2) The procedure is \emph{model-free}, or \emph{non-parametric} in that nothing is assumed on the parametric form of $y|x$.
(3) The proofs assume full knowledge of $F_x$, i.e., the joint distribution of predictors, marginalized over $y$.
(4) The method aims at good variable selection, not prediction.

We think of the method in \cite{SesiaGenehuntinghidden} as an adaptation of \cite{CandesPanninggoldmodelX2018} to genome-wide association studies (GWAS).
The differences between the two:
(1) The non-null variables in \cite{CandesPanninggoldmodelX2018} are those that belong to the minimal set that renders all others independent, i.e., the non-null is the Markov-Blanket of $x$ on $y$.
In \cite{SesiaGenehuntinghidden}, the non-null variables are those with non-null partial correlations.
(2) \cite{CandesPanninggoldmodelX2018} discusses a multivariate Gaussian model, while \cite{SesiaGenehuntinghidden} a hidden Markov model.
Each paper offers an sampling algorithm for sampling knockoffs from the assumed model.

The method of \cite{SesiaGenehuntinghidden} is also similar in flavor to \cite{BarberControllingfalsediscovery2015}, with the following differences:
(1) \cite{BarberControllingfalsediscovery2015} assume a linear generative model, so that the null is simply $H_j:\beta_j=0$.
(2) \cite{BarberControllingfalsediscovery2015} crucially assume $n>p$, and Gaussian distributed errors.
(3) \cite{BarberControllingfalsediscovery2015} infer conditional on $X$. They thus control the fixed-design FDR, and not the random-design.



\section{On the Problem Setup}

The problem setup in \cite{CandesPanninggoldmodelX2018} and \cite{SesiaGenehuntinghidden} deals with random design inference, in a non-parametric generative model.
We find this to be a very useful setup for screening problems:
It is consistent with the random designs typically found in observational studies, and it avoids the very-useful-yet-controversial linearity assumption.

A more surprising component of the problem setup is the knowledge of $F_x$, i.e., the joint distribution of predictors, marginalized over $y$.
Many authors would consider this an Oracle assumption, and given the difficulty of estimating joint distributions, an unrealistic one.
%[We, on the other hand, do not find this such a troubling assumption for the following reasons:
%(1) In many applications the dimensionality of $F_x$ can be reduced to a tractable one. The HMM assumption of \cite{SesiaGenehuntinghidden} is an example.
%(2) The preliminary results of \cite{CandesPanninggoldmodelX2018} show that FDR-control is quite robust to errors in $F_x$. We expect this matter to be further investigated in the future.]
The GWAS application, however, represents an ideal situation in which much is known about $F_x$, e.g.\ from the HapMap project \citep{Hapmap2003}.
Other areas of potential application include semi-supervised settings in which $F_x$ may be inferred from extra observations of $X$. 
In general high-dimensional data settings, however, $F_x$ has so many parameters that estimating it may turn out be a more difficult problem than estimating the conditional distribution of $y$ given $X$.
In all cases, robustness to errors in $F_x$ is crucial for the practical usefulness of the method, and we are happy to see the promising preliminary results of \cite{CandesPanninggoldmodelX2018}.
The pruning step the data examples in the paper suggests that high collinearity may adversely affect the performance of the method. We suspect that the higher the correlations in $F_x$, the more precisely $F_x$ should be known.
We expect this matter to be further investigated in the future.
[TODO: I wanted to be a bit more critical here. Agree with the new text?]


\section{Knockoffs as Pseudo-variables}
The idea of augmenting design matrices with random variables is not new.
It has been suggested many times, for the purposes of prediction, variable ranking, consistent support recovery, etc.
\cite{TusherSignificanceanalysismicroarrays2001} have already proposed the idea of permuting the original variables for FDR control on selected variables.
While intuitive and elegant, their algorithm did not have any provable guarantees, and implies marginal null and not conditional: $H_j:Cov[y,x_j]=0$, and not $H_j:Cov[y,x_j|x_{-j}]=0$.
Some more algorithms adding ``fake'', ``phony'', ``probes'' or ``pseudo variables'', are reviewed in \cite{GuyonIntroductionVariableFeature2003}.

Perhaps the most similar work is that of \cite{WuControllingVariableSelection2007}, which not only propose adding `pseudo-variables'' for the purpose of estimating the variable selection FDR, but also require two conditions very similar to the knockoff conditions.
\cite{WuControllingVariableSelection2007} require that:
``(A1) real unimportant variables and phony unimportant variables have the same probability of being selected on average'', and
``(A2) real important variables have the same probability of being selected whether or not phony variables are present''.
These two conditions cannot be satisfied according to \cite{WuControllingVariableSelection2007}, but they are clearly related to the \emph{pairwise exchangeability} and \emph{nullity condition} in \cite{SesiaGenehuntinghidden} and \cite{CandesPanninggoldmodelX2018}.
One may thus view the two knockoff conditions as a satisfiable version of A1 and A2.
To the credit of \cite{WuControllingVariableSelection2007} we quote their insights, which already hint at what will be later formalized in the knockoff conditions:
``Permutation produces pseudovariables that when appended to the real data create what
are essentially matched pairs. To each real variable there corresponds a pseudo variable with identical sample moments and also with preservation of correlations''.



\section{On Null Hypotheses and Invariants}
The problem of variable selection with error guarantees is not new.
Previously proposed algorithms include, for instance, Stability Selection \citep{MeinshausenStabilityselection2010}, SURE Screening \citep{fan2008sure}, BOLASSO \citep{bach2008bolasso}, Benjamini-Gavrilov \citep{Benjaminisimpleforwardselection2009}, and many more.
These procedures propose varying algorithms, with varying statistical guarantees in varying scenarios.
We do not review this literature for the sake of brevity.
We do, however, wish to discuss the matter of identifiability and estimability.
I.e., is the parameter well defined, and is the estimation problem well-posed?

When doing variable selection, one will always require some assumption to ensure that ``a good'' selection is well defined.
For this purpose a linear generative model is typically assumed.
In the linear generative case multicollinearity will render the problem non-identifiable.
To ensure identifiability in the fixed design, authors have proposed various conditions such as \emph{Sparse Eigenvalue}, \emph{Sparse Riesz Condition}, \emph{Neighbourhood Stability}, \emph{Irrepresentable Condition}, and \emph{Exact Recovery Criterion}.
See \citet[Sec 3.1.1]{MeinshausenStabilityselection2010} for a review.
In the random-design literature, identifiability is typically ensured with some restriction on the condition number of $Cov[x]$, or with some strong-convexity assumption on the risk surface.

How is identifiability ensured in the knockoff framework?
The crucial assumption is the knowledge of the distribution $F_x$, which avoids difficult to check sparsity conditions. 
The resulting invariance property (2) in the paper describes so-called null-invariant transformations \citep{Goeman2010}: the transformation $\textrm{swap}(S)$ leaves the joint distribution of the augmented data $(Y,X,\tilde X)$ untouched if $S$ is the set of true hypotheses, but disturbs it otherwise.
Known null-invariant transformations, e.g.\ permutations and rotations \citep{Langsrud2005}, existed so far only for null hypotheses about marginal association.
As far as we know, knockoffs are the first null-invariant transformations to be proposed for null hypotheses related to conditional association.
Seeing knockoffs as null-invariants opens the way for their more classical use, e.g.\ using many random knockoffs \citep{Hemerik2018}, or for controlling familywise error in the manner of \cite{Westfall1993}.
Since familywise error control is the norm in the field of GWAS, the latter would be a worthwile extension.


\section{The Question of causality}

The random-design knockoff framework does not deal with the linear generative case, so that null and alternative cannot be stated in terms of $\beta$ coefficients.
On the other hand, support recovery, unlike consistent estimation, does not require an identifiable model.
Errors, and thus FDR, can thus be defined without identifiable parameters.

Lacking a parametric generative model, null and alternative have a more information-theoretic flavor, and are stated in terms of dependence.
\citet{CandesPanninggoldmodelX2018} define the null using a Markov-Blanket, and \cite{SesiaGenehuntinghidden} use conditional independence.
This raises several questions:
Are these hypotheses consistent with the motivating problems?
Why this change in hypotheses?

Screening problems, such as GWAS, are good examples of variable selection:
The inference has no causal aspirations\footnote{Otherwise, stating hypotheses with \emph{Do-Calculus} \citep{pearl1995causal} may be more appropriate};
and the design is random.
We find that the conditional independence assumption is not quite consistent with the GWAS motivation.
To see this, consider two perfectly correlated SNPs.
One causal and the other not, but for the purpose of a screening study, both belong to the alternative.
Using the conditional independence hypotheses in \cite{SesiaGenehuntinghidden}, however, they will both belong to the null.
Using the Markov-Blanket hypotheses in \cite{CandesPanninggoldmodelX2018}, they will also belong to the null.
A similar concern was raised by J.T.\ Kent in the discussion of \cite{MeinshausenStabilityselection2010}.
It seems to us that for the purpose of screening, marginal hypotheses, such as the ones in \cite{TusherSignificanceanalysismicroarrays2001}, are more appropriate.
If the purpose were not screening for associations, but rather, causal inference, then clearly the definition of hypotheses, and errors, would have required a different language altogether.
The authors, however, postpone the causal question: ``By discovering which variables are important, scientists can design a more targeted follow-up investigation and hope to understand how certain factors influence an outcome.''
[TODO: do we have an opinion about this?]






\section{Future Research}
% robustness to $$F_x$$, robustness to true correlations, best statistic for power, algorithms to generate knockoffs, statistics for more than one knockoff,…

We find the knockoff framework to be quite exciting.
Not because it offers solutions to all possible difficulties, but on the contrary: because it sets the stage for many important research questions.
Are error guarantees robust to misspecification of $F_x$?
What test statistics have more power?
Knockoffs are not uniquely defined-- how to best generate them?
Do methods defined for other null-invariants, such as permutations, generalize to knockoffs easily?

\cite{DaiknockofffilterFDR2016}, \cite{JansonFamilywiseerrorrate2016}, \cite{ChenAnalysisKnockoffFilter2017}, \cite{ChenPseudoKnockoffFilter2017}, and certainly others, have already started to explore and extend the knockoff framework of \cite{BarberControllingfalsediscovery2015}.
We expect many such explorations in the upcoming future.



\section*{Acknowledgement}
The authors thank Prof. Yaakov Ritov, Dr. Aldo Solari, Dr. Livio Finos, and ... for fruitful discussions leading to this manuscript.


[TO DISCUSS; in relation to identifiab

\bibliographystyle{biometrika}
\bibliography{paper-ref}



\end{document}
